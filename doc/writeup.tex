\documentclass[11pt]{article}
\usepackage{amsmath}
\usepackage{amsfonts}
\usepackage{graphicx}
\setlength{\oddsidemargin}{0.0in}
\setlength{\evensidemargin}{0.0in}
\setlength{\topmargin}{-0.25in}
\setlength{\headheight}{0in}
\setlength{\headsep}{0in}
\setlength{\textwidth}{6.5in}
\setlength{\textheight}{9.25in}
\setlength{\parindent}{0in}
\setlength{\parskip}{2mm}



\begin{document}

\newenvironment{myindentpar}[1]%
{\begin{list}{}%
         {\setlength{\leftmargin}{#1}}%
         \item[]%
}
{\end{list}}

\title{JavaScript Profiler}
\date{October 18, 2013}
\author{Roy Adams and David Belanger}

\maketitle

\section{Overview}

	Our profiler takes as input a block of JavaScript code as a string, adds instrumentation code 
to all user defined functions, and then runs the code to collect profile data. The profiler may 
be called either as an imported library or via a simple HTML interface that allows the user to 
paste in JavaScript code he or she wants to profile. During a run of the input script, the profiler 
collects timing, call-path, and call frequency information which can then be accessed as a formatted output. 

\section{Design and Functionality}
	
Our approach to the problem involves two steps:
\begin{enumerate}
\item  The code to be profiled is parsed using the 
esprima parser (http://esprima.org/), which returns a syntax tree including line numbers and colums. This 
syntax tree is then traversed recursively and all function definitions, return statements, timing 
functions, and eval statements are detected and modified to include instrumentation code. Modifications 
are made directly to the syntax tree and so when the modifications are complete, the modified syntax 
tree is simply used to generate new code using the escodegen package \\(https://github.com/Constellation/escodegen).
\item The instrumented code is evaluated and profile information is gathered including call counts, function timing, call edges, and call paths. This second step can be done within our standalone profiler, 
profiler.html, or can be done within the user's application by replacing the code to be profiled 
with the instrumented code and importing profiling.js. In the first case, profile information is 
reported in the main page. In the second case, a plain-text version of
the same information is written to the browser console.
\end{enumerate}

\subsection{Code Modification}
	
	To modify the syntax tree we traverse it recursively, applying a modification function at each node.
All function declarations and function expressions are modified by adding code to the beginning and end 
of each function. All return statements are modified to run some instrumentation code prior to returning. 
In this way all user defined functions will execute profiling code regardless of how or from where 
they are called. As part of the instrumentation code, a function name is passed. If the user declares 
the function with a name, this name is retrieved during parsing and used when reporting profile information, 
otherwise the function is named anon\_line\_(line\_num)\_col\_(col\_num) where (line\_num) and (col\_num) are the line 
and column where the function is defined. An advantage of this strategy is that function statistics are 
agregated regardless of whether the function is assigned to a new variable. A downside is that the programmer 
must explicitly give the functions names is he/she want those names to show up in the profile. Also caught 
during parsing are any calls to setTimeout and setInterval. These are modified to 
indicate that the callback function passed is being called from the a timing function and indicate the 
line and column number of the timing function call. 

\subsection{Profiling}
	The profiler is defined completely in profiling.ts (which typescript then compiles to profiling.js) 
and consists of three main classes
\begin{itemize}
\item \textit{ProfilerFromSource} serves as an interface to the
  user. Its constructor modifies input javascript code using the
  techniques from the previous subsection, and it provides
  functionality for reporting profiling info both via plaintext
  reports and using the HighCharts library. The method startUp(), executes the
  instrumented JavaScript code, which may not mean that computation
  ends, due to the event-driven nature of many JavaScript
  applications. getReport() and getStringReport() return a summary of
  the current profiling information, to HighCharts images in wrapper
  html, or to a string. 

\item \textit{Profiler} stores and  maintains all global profile
  objects such as a call stack (an array of strings that gets pushed
  and popped), call paths, and a list of defined
  functions. It has logic for aggregating timing information and for
  producing a function's `self time,' where the time spent in
  instrumented subfunctions is subtracted off from the timing
  assessment of the function. Finally, the Profiler also defines various methods used in producing the
 reports that ProfilerFromSource returns.

\item \textit{Profile} is an object associated with every
  function. Note that we do not have one Profile per function call,
  but per function. The functions are indexed by name (either the name
  they have in the code, or their line and column numbers for
  anonymous functions). The Profiler keeps an associate array from
  function names to these Profile objects. When a function is
  executed, it sets profile= Profiler.getProfile() and increments timing
  information and invocation count information stored in it. It then
  calls profile.start(), does what the un-instrumented code did, and
  then calls profile.end(). These three functions, plus a member for
  the function to know its name, are the only things that need to be
  added to code in order instrument it. These start() and end()
  methods communicate with the Profiler object to push the function on
  and off the stack, log various performance metrics, etc. 

TODO: Roy: is
  there something special we do to handle setIntervals and setTimeouts
  when we push/pop them off the stack? is it related to this sentence: ` All functions passed to a timing function are wrapped in 
a function that pushes a dummy function indicating the type of timing function that is being set and its 
location in the original code.'?

\end{itemize}



\subsection{Profiler Output}
Profiler has methods called 'getReport()' and 'getStringReport()' that
return profiling info to be displayed to the user. In practice, we
wrap these in a setInterval so that profiling is interwoven with
execution, which is particularly important because of the event-driven
nature of javascript. The getReport() method makes nice-looking
figures using the third-party HighCharts library, and we render these
to the DOM. getStringReport() returns a string version of the same
info, which is useful if you want to profile code that interacts with
the DOM and will be dumping profiler info via console.log. 


For each instrumented function FOO, we report:
\begin{itemize}
\item \textit{Num Invocations}: Of all calls to
  instrumented functions, the percentage that were calls to FOO. 
\item  \textit{Average Time Below}: Percentage of total time in
  instrumented functions that we spent in a stack below FOO (i.e. the
  time spent by FOO and all of its children). 
\item \textit{Average Self Time}: Percentage of total time in
  instrumented functions that was spent within FOO and not in
  functions that it calls that are instrumented. 
\end{itemize}

We also report two more metrics:
\begin{itemize}
\item \textit{Top Parent-Child Call Frequencies}: The pairs of
parent-child functions that are called most frequently. We report the
top 5 pairs, but the call percentage is normalized across all observed
parent-child pairs.
\item \textit{Top Hot Paths from Root}: The stack traces all the way from the root
that are most frequent. Note that the presence of recursive functions
will make this report uninformative. However, it is robust to
asynchronous callbacks, like setTimeout(), that call themselves. This
will only be considered as a single depth in the stack. Each call
won't increase its depth. 
\end{itemize}

Note that all our timing metrics may have high variance and limited
reproducibility due to the event-driven nature of many javascript
applications and the fact that different asynchronous callbacks can be
interwoven arbitrarily. 

In addition, note that these metrics, both for timing, and
numbers of invocations, are only aggregated across the functions that
have been instrumented. It is possible, for example, that there are
functions from third-party libraries that are called more than any
profiled function. Similarly, when we report the self-time for a
function, we do not subtract off the time spent in non-profiled
functions from third parties. 


\subsection{Asynchronous Callbacks and Arbitrarily-Deep Stacks}
	By directly modifying function definitions, asynchronous callbacks do not interfere with 
the profiling. Any asynchronous callbacks will trigger the profiling code like any other call. 
Timing functions, such as setTimeout and setInterval, have also been modified so 
that in the profile report, the location of the original call to setTimeout or setInterval 
is reported as the calling function. In order to profile asynchronously called functions, we create
our own setInterval function which reports the most current profile stats every second. In this 
way, a user can set a program with asynchronous callbacks running and check back periodically 
to see what the current state of the profile is or interact with the program to trigger asynchronous 
callbacks and check the effect on the profile. 
TODO: David discusses Deep stacks

\subsection{User Interfaces}
There are two main ways to interface with the profiler. 

\begin{itemize}
\item For simple JavaScript code, particularly code that doesn't
  interract with the DOM, we recommend using  profiler.html, which can
  be run in any browser, allows the user to paste in any JS code and
  getting profiling information. It parses the code and then evaluates the modified code and updates profile statistics every second. 
This is obviously not intended as a server side interface as simply evaling arbitrary code is 
about as unsecure as it is possible to be. Another caveat to this interface is that the JavaScript 
code can only have very limited, if any, interaction with the DOM without messing up the profiling 
environment. Much JavaScript is heavily interactive so this will not be the interface of choice 
for many situations. It is instead intended as an interface to quickly and simply profile standalone 
code fragments.

\item  An alternative use case is where the user takes his or her
  JavaScript code and uses our tools to produce new instrumented code
  that they can swap out in their application. In this mode, it writes
  profiling info to console, so the idea is that it won't interfere
  with the functionality of code that interracts with the DOM or
  requires network traffic. To generate instrumented code, the user may 
open gen\_prof\_code.html and paste in any code as before. The code is then modified and 
the modified code is output to the window. The user then copies this new code to a new JS file 
to be used in place of the original file. To run the profiler, the user simply imports profiling.js
and replaces the original code file with the new modified code and
runs his/her html application. The profiler will then periodically write the current profile statistics to the console.
This method allows the user to run the profiler in any html application 
and allows the user to interact with the DOM in any way he/she wants (subject to not overwriting the 
GlobalProfile object).
\end{itemize}

\section{Testing}
We provide a suite of tests in the scripts/testScripts
directory. These are a selection of the tests from the Computer
Language Benchmark Game (benchmarksgame.alioth.debian.org). The only
modification is that we defined a function doTest() at the bottom of
all of them that acts as a driver function and returns a string
representing the results of the test's computation. The basic idea for
the test suite is that for every file we do the following:
\begin{enumerate}
\item Run the original code, by calling eval on the code and then
  calling doTest();
\item Modify the code to be instrumented for profiling, eval this, and call
  doTest().
\item Assert that these two procedures returned the same results. 
\end{enumerate}
The goal of this test suite is to ensure that instrumentation doesn't
change the output of javascript code. It does not assert that the
instrumented code interracts identically with the DOM, though it
should, and this is something that we tested with simple examples on
our own. Note, again, that there are security issues with running this
test suite, since it calls eval on the test scripts. 

We also provide a directory scripts/tests, which contain analogous
versions of these scripts, but that execute computation, when sourced
as a script, and don't require running doTests(). These can be used as
examples to be pasted into profiler.html.

\textit{Note for running the test suite in Chrome: }\\
The tests are saved on the file system and we have not made them
available through http requests. Therefore you need to run a jquery
request that loads local files. Chrome has safeguards against this. To
turn off these safeguards, close chrome and re-launch chrome from the command
line with the option '--allow-file-access-from-files. '

\section{Third-Party Packages Used}
\begin{enumerate}
\item jquery - Allows cleaner access to the DOM and provides a means
  to get scripts directly from the local filesystem. Needed for highcharts
  and the test suite. (http://jquery.com/)
\item esprima - Parses JavaScript code and returns a syntax tree. (http://esprima.org/)
\item escodegen - Generates executable JavaScript code from an esprima style syntax tree. \\(https://github.com/Constellation/escodegen)
\item highcharts - JS library that produces formatted plots. (http://www.highcharts.com/)
\end{enumerate}
\end{document}

\section{Compilation from TypeScript Source}
We compile scripts/profiling.ts to scripts/profiling.js
