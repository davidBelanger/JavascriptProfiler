\documentclass[11pt]{article}
\usepackage{amsmath}
\usepackage{amsfonts}
\usepackage{graphicx}
\setlength{\oddsidemargin}{0.0in}
\setlength{\evensidemargin}{0.0in}
\setlength{\topmargin}{-0.25in}
\setlength{\headheight}{0in}
\setlength{\headsep}{0in}
\setlength{\textwidth}{6.5in}
\setlength{\textheight}{9.25in}
\setlength{\parindent}{0in}
\setlength{\parskip}{2mm}



\begin{document}

\newenvironment{myindentpar}[1]%
{\begin{list}{}%
         {\setlength{\leftmargin}{#1}}%
         \item[]%
}
{\end{list}}

\title{JavaScript Profiler}
\date{October 18, 2013}
\author{Roy Adams and David Belinger}

\maketitle

\section{Overview}

	Our profiler takes as input a block of JavaScript code as a string, adds instrumentation code 
to all user defined functions, and then runs the code to collect profile data. The profiler may 
be called either as an imported library or via a simple HTML interface that allows the user to 
paste in JavaScript code he or she wants to profile. During a run of the input script, the profiler 
collects timing, call-path, and call frequency information which can then be accessed as a formatted output. 

\section{Design and Functionality}
	
	Our approach to the problem involves two steps. First, the code to be profiled is parsed using the 
esprima parser which returns a syntax tree including line numbers and colums. This 
syntax tree is then traversed recursively and all function definitions, return statements, timing 
functions, and eval statements are detected and modified to include instrumentation code. Modifications 
are made directly to the syntax tree and so when the modifications are complete, the modified syntax 
tree is simply used to generate new code using the escodegen package. In the second 
main step, this code is evaluated and profile information is gathered including call counts, function 
timing, call edges, and call paths. This second step can be done within our standalone profiler, 
profiler.html, or can be done within the user's application by replacing the code to be profiled 
with the instrumented code and importing profiling.js. In the first case, profile information is 
reported in the main page. In the second case, a popup window is opened which reports the same information.

\subsection{Code Modification}
	
	To modify the syntax tree we traverse it recursively, applying a modification function at each node.
All function declarations and function expressions are modified by adding code to the beginning and end 
of each function. All return statements are modified to run some instrumentation code prior to returning. 
In this way all user defined functions will execute profiling code regardless of how or from where 
they are called. As part of the instrumentation code, a function name is passed. If the user declares 
the function with a name, this name is retrieved during parsing and used when reporting profile information, 
otherwise the function is named anon\_line\_(line\_num)\_col\_(col\_num) where (line\_num) and (col\_num) are the line 
and column where the function is defined. An advantage of this strategy is that function statistics are 
agregated regardless of whether the function is assigned to a new variable. A downside is that the programmer 
must explicitly give the functions names is he/she want those names to show up in the profile. Also caught 
during parsing are any calls to setTimeout and setInterval. These are modified to 
indicate that the callback function passed is being called from the a timing function and indicate the 
line and column number of the timing function call. 

\subsection{Profiling}

	The profiler is defined completely in profiling.ts (which typescript then compiles to profiling.js) 
and consists of three main classes: ProfileFromSource, Profiler, and Profile. The top level class 
is ProfileFromSource which serves as an interface to the user. It has three main functions, a constructor 
which modifies the input code, startUp which runs the modified code and collects profile information, 
and getReport which compiles the profile information and returns a formatted string. Underneath this 
is the Profiler class which does most of the leg-work in our profiler. The Profiler class stores and 
maintains all global profile objects such as the call stack, call paths, and a list of defined functions. 
Finally, the Profile class maintains information about a specific function such as average time spent in 
the function and number of invocations. When a user defined function, call it Foo, is called it calls the Profiler member 
function getProfile which searches the list of profile functions already created for a matching profile 
and creates a new profile if one is not found. Foo then calls Profile.start which pushes Foo onto the 
call stack and begins timing. Finally, just prior to returning, Foo calls Profile.end which pops Foo from 
the call stack and does some timing calculations. All functions passed to a timeing function are wrapped in 
a function that pushes a dummy function indicating the type of timing function that is being set and its 
location in the original code.

\subsection{Profiler Output}
Profiler has methods called 'getReport()' and 'getStringReport()' that
return profiling info to be displayed to the user. In practice, we
wrap these in a setInterval so that profiling is interwoven with
execution, which is particularly important because of the event-driven
nature of javascript. The getReport() method makes nice-looking
figures using the third-party HighCharts library, and we render these
to the DOM. getStringReport() returns a string version of the same
info, which is useful if you want to profile code that interacts with
the DOM and will be dumping profiler info via console.log. 

These reports have the following info:
\begin{itemize}
\item TODO
\item  TODO
\end{itemize}


\subsection{Asynchronous Callbacks and Arbitrarily-Deep Stacks}
	By directly modifying function definitions, asynchronous callbacks do not interfere with 
the profiling. Any asynchronous callbacks will trigger the profiling code like any other call. 
Timing functions, such as setTimeout and setInterval, have also been modified so 
that in the profile report, the location of the original call to setTimeout or setInterval 
is reported as the calling function. In order to profile asynchronously called functions, we create
our own setInterval function which reports the most current profile stats every second. In this 
way, a user can set a program with asynchronous callbacks running and check back periodically 
to see what the current state of the profile is or interact with the program to trigger asynchronous 
callbacks and check the effect on the profile. 
TODO: David discusses Deep stacks

\subsection{User Interfaces}
	There are two main ways to interface with the profiler. The first and simplest way is to use 
profiler.html which can be run in any browser. This allows the user to paste in any JS code. It 
parses the code and then evaluates the modified code and updates profile statistics every second. 
This is obviously not intended as a server side interface as simply evaling arbitrary code is 
about as unsecure as it is possible to be. Another caveat to this interface is that the JavaScript 
code can only have very limited if any interaction with the DOM without messing up the profiling 
environment. Much JavaScript is heavily interactive so this will not be the interface of choice 
for many situations. It is instead intended as an interface to quickly and simply profile standalone 
code fragments.
	We consider the second interface to be the typical use case. To profile code, the user may 
open gen\_prof\_code.html and paste in any code as before. The code is then modified and 
the modified code is output to the window. The user then copies this new code to a new JS file 
to be used in place of the original file. To run the profiler, the user simply imports profiling.js
and replaces the original code file with the new modified code and runs his/her html application 
as normal. The profiler will then open a popup that report profile information every second while the
original program is running. This method allows the user to run the profiler in any html application 
and allow the user to interact with the DOM in any way he/she wants (subject to not overwriting 
window.report which is used to pass information to the report window).

\section{Testing}
We provide a suite of tests in the scripts/testScripts
directory. These are a selection of the tests from the Computer
Language Benchmark Game (benchmarksgame.alioth.debian.org). The only
modification is that we defined a function doTest() at the bottom of
all of them that acts as a driver function and returns a string
representing the results of the test's computation. The basic idea for
the test suite is that for every file we do the following:
\begin{enumerate}
\item Run the original code, by calling eval on the code and then
  calling doTest();
\item Modify the code to be instrumented for profiling, eval this, and call
  doTest().
\item Assert that these two procedures returned the same results. 
\end{enumerate}
The goal of this test suite is to ensure that instrumentation doesn't
change the output of javascript code. It does not assert that the
instrumented code interracts identically with the DOM, though it
should, and this is something that we tested with simple examples on
our own. Note, again, that there are security issues with running this
test suite, since it calls eval on the test scripts. 

\textit{Note for running the test suite in Chrome: }\\
The tests are saved on the file system and we have not made them
available through http requests. Therefore you need to run a jquery
request that loads local files. Chrome has safeguards against this. To
turn off these safeguards, close chrome and re-launch chrome from the command
line with the option '--allow-file-access-from-files. '

\section{Third-Party Packages Used}
\begin{enumerate}
\item jquery - Allows cleaner access to the DOM. Needed for highcharts. (http://jquery.com/)
\item esprima - Parses JavaScript code and returns a syntax tree. (http://esprima.org/)
\item escodegen - Generates executable JavaScript code from an esprima style syntax tree. \\(https://github.com/Constellation/escodegen)
\item highcharts - JS library that produces formatted plots. (http://www.highcharts.com/)
\end{enumerate}
\end{document}