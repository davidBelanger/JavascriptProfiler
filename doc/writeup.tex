\documentclass[11pt]{article}
\usepackage{amsmath}
\usepackage{amsfonts}
\usepackage{graphicx}
\setlength{\oddsidemargin}{0.0in}
\setlength{\evensidemargin}{0.0in}
\setlength{\topmargin}{-0.25in}
\setlength{\headheight}{0in}
\setlength{\headsep}{0in}
\setlength{\textwidth}{6.5in}
\setlength{\textheight}{9.25in}
\setlength{\parindent}{0in}
\setlength{\parskip}{2mm}



\begin{document}

\newenvironment{myindentpar}[1]%
{\begin{list}{}%
         {\setlength{\leftmargin}{#1}}%
         \item[]%
}
{\end{list}}

\title{JavaScript Profiler}
\date{October 18, 2013}
\author{Roy Adams and David Belinger}

\maketitle

\section{Overview}

	Our profiler takes as input a block of JavaScript code as a string, adds instrumentation code 
to all user defined functions, and then runs the code to collect profile data. The profiler may 
be called either as an imported library or via a simple HTML interface that allows the user to 
paste in JavaScript code he or she wants to profile. During a run of the input script, the profiler 
collects timing, call-path, and call frequency information which can then be accessed as a formatted output. 

\section{Design}
	
	The profiler is defined completely in profiling.ts (which typescript then compiles to profiling.js) 
and consists of three main classes: ProfileFromSource, Profiler, and Profile. The top level class 
is ProfileFromSource which serves as an interface to the user. It has only three functions, a constructor 
which modifies the input code, startUp which runs the modified code and collects profile information, 
and getReport which compiles the profile information and returns a formatted string. Underneath this 
is the Profiler class which does most of the leg-work in our profiler. The Profiler class stores and 
maintains all global profile objects such as the call stack, call paths, and a list of defined functions. 
Finally, the Profile class maintains information about a specific function such as average time spent in 
the function and number of invocations.

\section{Interface}

	To use the profiler as a library 

\section{Discussion}

\end{document}